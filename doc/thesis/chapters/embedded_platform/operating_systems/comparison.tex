\subsection{Comparison}
\label{subsec:embedded_platform:operating_systems:comparison}
Table \ref{tab:compare_os} shows a quick overview of the main functions and properties of the two \acrlong{os}

\begin{table}[hb]
  \caption{Compare PYNQ and Petalinux}
  \label{tab:compare_os}
  \centering
  \begin{tabular}{lll}
    \toprule
    \textbf{} & \textbf{PYNQ} & \textbf{Petalinux} \\
    \midrule
    Kernelversion & 4.19.0 & 4.19.0 \\ % cite: , cite https://www.xilinx.com/support/answers/72293.html
    Basic distribution & ubuntu & none \\ % cite: https://www.xilinx.com/publications/events/xtd/05_Xilinx_Pynq_framework_to_support_Python_and_IIOT.pdf
    Supporting Baumer Camera & (yes) & no \\ % cite https://www.baumer.com/medias/__secure__/Baumer_Digital-Industrial-Cameras_EN_20200514_BR_11228587.pdf?mediaPK=8828642721822
    Wifi function & yes & (yes) \\
    Package manager & \acrshort{apt} & none, installable \\
    dpu initialization & with overlays at runtime & while booting \\ % cite: , cite: https://pynq.readthedocs.io/en/v2.5.1/pynq_overlays.html
    \bottomrule
  \end{tabular}
\end{table}
\todo[inline]{find source to pynq kernel version, hard to find}
\todo[inline]{find source to petalinux flash moment}

PYNQ is based on Ubuntu.
Because Baumer supports Ubuntu, the Baumer camera should work with this \acrlong{os}.
However, installing the official .deb package does not work.
The workaround is to use the installing package for a Rock64 board which Baumer distributed for a while.
A Rock64 development board consists of a Quad-Core ARM Cortex A53 processor just as the Ultrascale+ family.
Further mysqlclient and pq5 has to be installed to use the Baumer camera.

Altough, both \acrshort{os} officially support wifi petalinux has connecting problems due to a faulty root filesystem.

On PYNQ the \acrfull{apt} is installed.
This is the default package manager on many debian distributions.
Recipes can be added as described in section \ref{subsec:embedded_platform:operating_systems:petalinux} to use a package manager on petalinux.

