\section{Model Deployment}
\label{sec:embedded_platform:model_deployment}
Xilinx provides functions to prepare the in chapter \ref{ch:training_of_the_cnn} deployed model for the \acrshort{dpu}.
The preparation is done in two main steps:
\begin{enumerate}
	\item Quantization
	\item Compilation
\end{enumerate}
Xilinx makes it easy to use those functions by using docker.
Docker makes it possible to run an application in an isolated environment called container.
The Xilinx images to create a container are distributed via DockerHub.
An image is pulled automatically when a command like
\begin{lstlisting}[style=bash, caption={}, label=lst:pull_img]
  docker run "xilinx/vitis-ai:1.1.56"
\end{lstlisting}
is executed for the first time \cite{docker_overview}.
The functions to quantize and compile are installed in the xilinx/vitis-ai image in binary form.
The process showed in figure \ref{fig:petalinux_workflow} is valid apart from building the Vitis Platform.

\subsection{Quantize}
\label{subsec:embedded_platform:model_deployment:quantize}
The function to quantize a Tensorflow model is named vai\_q\_tensorflow, where vai stands for Vits AI and q means quantize.
Quantize a Caffe model is possible with the vai\_q\_caffe function.
The quantizer has the following parameters:
\begin{itemize}
	\item quantize
	\item dump
\end{itemize}

\paragraph{Quantize}
The Vitis AI quantizer takes a floating-point model as input (prototxt and caffemodel for the
Caffe version, and frozen GraphDef file for the TensorFlow version).
First the useless nodes are removed and the batchnorms are folded.
In a next step the quantization is applied.
After quantization the weights, biases and activations have a given bit width, like for example eight bit.

A calibration image dataset is required to check the quality of the quantized neural network.
This dataset contains between \numrange{100}{1000} unlabeled frames \cite{vitis_ai_user_guide}.
The output of the quantization is a \acrshort{dpu} deployable model named deploy\_model.pb for a tensorflow model and a deploy.prototxt/deploy.caffemodel for caffe.

Listing \ref{lst:quantize} shows an example how the quantize command could look like.

\begin{lstlisting}[style=bash, caption={Quantize command}, label=lst:quantize]
  vai_q_tensorflow quantize \
  --input_frozen_graph frozen_graph.pb \
  --input_nodes x \
  --input_shapes ?,256,320,3 \
  --output_nodes Identity \
  --input_fn input_fn.calib_input \
  --calib_iter 50 \
  --output_dir build \
  --method 1
\end{lstlisting}

The input node names as well as the output node names are set when generating the frozen graph.
These are the first and last entries when the layers are listed.
The parameter input\_frozen\_graph is the path and name of the frozen graph generated in section \ref{sec:architecture}. \todo[inline]{Correct link to section}
The input\_nodes defines the first layer to be quantized and the last layer is set by the output\_nodes.
The input\_fn is a python script which processes the calibration images as it is done during inference.
Amongst others the path to the calibration dataset is set in the python script.
Calib\_iter defines the number of iterations.
100 is the default number but 50 turned out to be quite enough.
Method can be set either to 0 or 1.
1 is the default and sets the min-diff method.
When method is set to 0 no values are saturated during quantization which makes it sensitive to outliers.

\paragraph{Dump}
It is possible to dump the quantized model for example to debug or for better understanding of the deployed \acrshort{cnn}.
The difference between the listing \ref{lst:quantize} and the dump command is the batch size which is now called max\_dump\_batches and can be set to one.
The second difference is the input\_frozen\_graph should no be the quantized output.
There is one more parameter called dump\_float.
If dump\_float is set to 1 the float weights and activations are dumped as well.
The default is 0.
Furthermore calling the dump function is done by vai\_q\_tensorflow dump.

\paragraph{Prune}
Most neural network have significant redundancy to achieve a certain accuracy.
The redundancy comes from over-parameterized models.
Pruning is a process of removing some of those weights while keeping the accuracy loss as low as possible \cite{pruning_overview}.
Pruning a model needs a lot of hardware resources and time.
Although pruning is part of the quantize procedure, it is packed in the Vitis AI optimizer.
To get the optimizer it is needed to contact the Xilinx support team.
Therefore, pruning is not applied in this project.

\subsection{Compile}
\label{subsec:embedded_platform:model_deployment:compile}
According to the Vitis AI quantize tool the compile command is named vai\_c\_tensorflow.
Goal of the compiler is to map the network model into a highly optimized \acrshort{dpu} instruction sequence.

Start compiling can be done like shown in listing \ref{lst:compile}.

\begin{lstlisting}[style=bash, caption={Compile command}, label=lst:compile]
  vai_c_tensorflow \
  --arch arch.json \
  --frozen_pb deploy_model.pb \
  --output_dir build \
  --net_name fhnw_toys_0 \
  --options "{'mode': 'normal'}"
\end{lstlisting}

The \texttt{arch.json} contains information about the hardware in a key value pair format.
To define the \acrshort{dpu} as edge \acrshort{dpu} the key \texttt{target} is set to \texttt{dpuv2}.
Furthermore, the \texttt{cpu\_arch} is set to \texttt{arm64} for Ultrascale+ \acrshort{mpsoc}.
The compiler also needs the built \acrshort{dpu} information as a \acrfull{dcf}.
The \texttt{dpu.hwh} file contains all information to generate a \acrshort{dcf}.
Xilinx provides a function called \texttt{dlet} in the docker image.
By executing the command
\begin{lstlisting}[style=bash, caption={}, label=lst:dlet]
  dlet -f dpu.hwh
\end{lstlisting}
a file called \texttt{dpu-dd-mm-yyyy-hh-mm.dcf} is created where dd-mm-yyyy-hh-mm is the timestamp when the \texttt{dpu.hwh} file was created \cite{vitis_ai_user_guide}.

The whole \texttt{arch.json} file can look like this:
\begin{lstlisting}[style=bash, caption={}, label=lst:arch_json]
  {
  	"target": "dpuv2",
  	"dcf": "'dpu-01-01-2000-00-01.dcf'",
  	"cpu_arch": "arm64"
  }
\end{lstlisting}

Due to a programming error of Xilinx the net name must not end with an s.
The Python script which loads the overlay removes everything from the end of the netname until there is no s or \_0.
\todo[inline]{what does it remove exactly?}

The \acrshort{dpu} mode can be set either to normal, profile or debug mode.
In normal mode the application can get the best performance.
When using the profile mode, output is printed to the console.
If debug mode is selected, raw data for each \acrshort{dpu} computation node will be dumped.

After the compilation is done a file named \textit{netname}.elf is stored in the defined output directory.
In the example listing \ref{lst:compile} it would be \texttt{fhnw\_toys\_0.elf}.