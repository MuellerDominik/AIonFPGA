\section{DPU}
\label{sec:embedded_platform:dpu}

The Xilinx \acrfull{dpu} is developed to run convolutional neural networks on an \acrshort{fpga} of the Zynq UltraScale+ or Zynq-7000 family.
\acrshort{dpu}v1 and \acrshort{dpu}v3 are for cloud computing on Alveo cards and \acrshort{dpu}v2 for Zynq devices. 
Due to parallelization of the calculations in a neural network, the inference can be accelerated considerably.
The \acrshort{dpu} is configured by an AXI slave interface and accesses instructions by a AXI master interface.
It can be configured to make best use of the size of the \acrshort{fpga}.
The encrypted \acrshort{rtl} design files are available on Github.
To use the \acrshort{dpu}, a device driver is required \cite{dpu_product_guide}.
This driver is included in the Xilinx Vitis AI development kit \cite{dpu_product_guide_v3_2}.

\subsection{Configuration}
\label{subsec:embedded_platform:dpu:configuration}
\paragraph{Number of \acrshortpl{dpu}}
The number of \acrshortpl{dpu} is set in the \texttt{prj\_config} configuration file with nk=dpu\_xrt\_top:x.
The variable x stands for the number of \acrshortpl{dpu}.
Depending on the Ultrascale+ device, one to three \acrshort{dpu} cores can be implemented.
For example on a ZU02 to ZU5 device, only one core is supported.
On a ZU7 device it is possible to implement two cores while on a ZU9 even three cores are feasible.

\paragraph{Clock}
The clock frequency is set in the \texttt{prj\_config} file as well.
For the ZU3 device, the maximum clock frequency is \SI{370}{MHz}.
Because of a hardware design mistake on the Ultra96-V2 board, the maximum frequency lies at \SI{300}{MHz} with PYNQ and \SI{150}{MHz} with Petalinux.
On higher frequencies the \acrfull{pmic} reboots the hole system.
It is possible to fix this issue, but it needs both a hardware and a software change \cite{pmic_issue}.
The frequency can be set  by the setting the variable  freqHz=150000000:DPUCZDX8G\_1.aclk.
Another way is using the predefined frequencies from Xilinx.
Table \ref{tab:frequencies_ids} shows the possibilities.

\begin{table}[hb]
  \caption{Ids and their Frequencies}
  \label{tab:frequencies_ids}
  \centering
  \begin{tabular}{ll}
    \toprule
    \textbf{ID} & \textbf{Frequency} \\
    \midrule
    0 & \SI{150}{MHz} \\
    1 & \SI{300}{MHz} \\
    2 & \SI{75}{MHz} \\
    3 & \SI{100}{MHz} \\
    4 & \SI{200}{MHz} \\
    5 & \SI{400}{MHz} \\
    6 & \SI{600}{MHz} \\
    \bottomrule
  \end{tabular}
\end{table}

To use the ID, replace freqHz with id and 15000000 with 0.
Besides this clock the ap\_clk\_2 must be set as well.
This one should have double frequency of the aclk and is the one which should not exceed \SI{300}{MHz}.

\paragraph{\acrshort{dpu} Architecture}
The following configurations are set in the \texttt{dpu\_conf.vh} file.
The \acrshort{dpu} IP can be configured with different convolution architectures.
They are related to the parallelism of the convolution unit.
On a ZU3EG the best results can be achieved with a B2304 architecture.
2304 are the maximum operatings per clock.
Therefore, a higher number means higher throughput.
Table \ref{tab:arch_parallelism} shows all available architectures and the correlation to the parallelism.

\begin{table}[hb]
  \caption{Parallelism for Different Convolution Architectures \cite{dpu_product_guide}}
  \label{tab:arch_parallelism}
  \centering
  \begin{tabular}{lllll}
    \toprule
    \textbf{Architecture} & \textbf{Pixel} & \textbf{Input Channel} & \textbf{Output Channel} & \textbf{Peak Ops} \\
     & \textbf{Parallelism} & \textbf{Parallelism} & \textbf{Parallelism} & \textbf{(ops per clock)} \\
    \midrule
    B512 & 4 & 8 & 8 & 512 \\
    B800 & 4 & 10 & 10 & 800 \\
    B1024 & 8 & 8 & 8 & 1024 \\
    B1152 & 4 & 12 & 12 & 1150 \\
    B1600 & 8 & 10 & 10 & 1600 \\
    B2304 & 8 & 12 & 12 & 2304 \\
    B3136 & 8 & 14 & 14 & 3136 \\
    B4096 & 8 & 16 & 16 & 4096 \\
    \bottomrule
  \end{tabular}
\end{table}

\paragraph{RAM Usage}
The RAM Usage can be set high or low.
If the chip does not possess UltraRAM the on-chip block RAM is used and the UltraRAM must be disabled.
One block has a size of \SI{36}{kB} and for the B2304 architecture 167 such blocks are used for RAM low and 211 for RAM high.
If UltraRAM is available, several other defines need to be set.
The listing \ref{lst:config_dpu} shows an example, how to do it with a B4096 architecture.
The UltraRAM numbers can be found in the Zynq \acrshort{dpu} \acrshort{ip} product guide.
The ZU3EG device is in possession of 216 BRAM and no UltraRAM.
Because the RAM for the \acrshort{pl} is not used for any other calculation, RAM high can be activated and UltraRAM should be disabled.
Practical tests showed throughput time reductions of about 20\% with RAM high compared to RAM low.
\todo[inline]{chamer so öppis do ine schriibe?}

\paragraph{Channel Augmentation}
This feature improves the efficiency of the \acrshort{dpu} when the number of input channels is lower than the available channel parallelism.
Activating this function costs extra logic resources.
In this CNN there are three input channels and therefore, channel augmentation is enabled.

\paragraph{Depthwise Convolution}
In standard convolution, the operations are performed for each input channel separately with one specific kernel and the results are combined over all channels.
In depthwise separable convolution, the operation is performed in two steps: depthwise convolution and pointwise convolution.
Enabling  depthwise convolution allows for more flexibility with the \acrshort{cnn} architecture.

\paragraph{AveragePool}
Average pooling can be done on the \acrshort{dpu} or not.
The supported sizes goes from $2\cdot2$ to $8\cdot8$, with only square sizes supported.

\paragraph{ReLU Type}
By default the ReLu and ReLU6 function are supported.
It is possible to include the LeakyReLU function as well with low resource usage.

\paragraph{Softmax}
The softmax function can be done in hardware or in software.
Implemented on hardware, the function is about 160 times faster than a software implementation.
The function needs approximately \SI{10000}{\acrshortpl{lut}}, four block RAMs and 14 DSPs.
If the the resources run out the \acrlong{dnnc} implements the function independently in software.
The softmax function is not available in Zynq-7000 devices \cite{dpu_product_guide}. 

\paragraph{Configuration File}
With all the configuration settings, the \texttt{dpu\_config.vh} file for a ZU3EG device could look like this:

\begin{lstlisting}[style=bash, caption={Configure \acrshort{dpu}}, label=lst:config_dpu]
  `define B2304
  `define URAM_DISABLE 

  // Config URAM
  `ifdef URAM_ENABLE
    `define def_UBANK_IMG_N          5
    `define def_UBANK_WGT_N          17
    `define def_UBANK_BIAS           1
  `elsif URAM_DISABLE
    `define def_UBANK_IMG_N          0
    `define def_UBANK_WGT_N          0
    `define def_UBANK_BIAS           0
  `endif
  `define RAM_USAGE_HIGH
  `define CHANNEL_AUGMENTATION_ENABLE
  `define DWCV_ENABLE
  `define POOL_AVG_ENABLE
  `define RELU_LEAKYRELU_RELU6
  `define DSP48_USAGE_LOW
\end{lstlisting}

\subsection{Building}
\label{subsec:embedded_platform:dpu:building}

The \texttt{DPU-PYNQ} repository from Xilinx which is used to patch PYNQ can be used to build the \acrshort{dpu} as well.
It needs to be cloned to the host computer.
By cloning the repository the \texttt{Vitis-AI} repository is cloned as a submodule.
It contains the \texttt{DPU-TRD} folder to build a \acrshort{dpu}.
In a next step the created configuration files need to be copied to the \texttt{boards/Ultra96} directory.
Now, the Xilinx tools Vitis and \acrshort{xrt} must be sourced as shown in listing \ref{lst:source_tools}.
The makefile checks if all required tools are installed and sourced and then starts Vivado.
During the building process which can take more than an hour the bitstream is generated. 
After running the makefile in the \texttt{boards} directory with the parameter BOARD=Ultra96, the files \texttt{dpu.bit}, \texttt{dpu.hwh} and \texttt{dpu.xclbin} are created and located in the same directory as the configuration files where stored.
Those files can be copied to the board and the customized \acrshort{dpu} is ready to use.

When working with Petalinux the generated \texttt{BOOT.BIN} is of interest as well.
This file is located in the \texttt{boards/Ultra96/binary\_container\_1/sd\_card} directory.
The \texttt{BOOT.BIN} is responsible to load the \acrshort{dpu} on boot with the \texttt{.xclbin} file.
The \texttt{.xclbin} file file should be located at \texttt{/usr/lib} on the board.
Using PYNQ the \acrshort{fpga} can be programmed while starting the application.
