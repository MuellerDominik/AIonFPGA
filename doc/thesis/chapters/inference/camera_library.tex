\section{Camera Library}
\label{sec:inference:camera_library}
% \todo[inline]{needs some improvements}

% The camera library shares a great
The camera library is very similar to the camera application created during the previous project.
Therefore, only the differences and the various functions are described in this section.

% Differences:
% - major difference = source code of the camera interface is compiled to a shared library and not to an application
% - own buffer implemenetd
% - glitch will not decrement the buffer size (introduction of a throw_bgn local variable)
% - also glitch will not be a false positive for an application polling the global variable
The main difference between the old camera application and the new camera library is the way they are compiled.
Previously, the source code of the camera interface was compiled to an application, whereas now it is compiled to a shared library (\texttt{libcamera.so}).

Furthermore, the various state and configuration variables are now globally declared.
This allows an application which uses the shared library to poll these global variables and act upon a change.

The necessary memory for the images is manually allocated and then passed to the constructor of the Baumer \texttt{BGAPI2::Buffer::Buffer}.
This leads to a decrease in memory usage of about \SI{66}{\percent}. 

The glitch detection was revised.
Removed glitches (single frame changes) are no longer reducing the size of the buffer.
It has also been ensured that when a glitch occurs, that the global variable \texttt{throw\_bgn} is not set to \texttt{true}.
Otherwise, an application polling the \texttt{throw\_bgn} variable would experience a false positive.

\subsection{Functions}
\label{subsec:inference:camera_library:functions}

Besides the main functions, the camera library also features several accessor functions to access global variables and mutator functions to control camera parameters.
% The camera library consists of accessor and mutator function.

The main functions:

\begin{enumerate}
  \item \texttt{initialize} -- used to initialize the camera
  \item \texttt{reset\_global\_variables} -- used to reset the global variables
  \item \texttt{start\_acquisition} -- used to start the image acquisition
  \item \texttt{terminate} -- used to terminate the camera
\end{enumerate}

Accessor functions, used to access global variables:

\begin{enumerate}
  \item \texttt{get\_frame\_ptr} -- used to get a specific frame pointer
  \item \texttt{get\_throw\_bgn} -- used to get the value of the global variable \texttt{throw\_bgn}
  \item \texttt{get\_throw\_end} -- used to get the value of the global variable \texttt{throw\_end}
  \item \texttt{get\_throw\_bgn\_idx} -- used to get the value of the global variable \texttt{throw\_bgn\_idx}
  \item \texttt{get\_throw\_end\_idx} -- used to get the value of the global variable \texttt{throw\_end\_idx}
\end{enumerate}

Mutator functions, used to control camera parameters:

\begin{enumerate}
  \item \texttt{set\_frame\_rate} -- used to modify the frame rate
  \item \texttt{set\_buff\_size} -- used to modify the buffer size
  \item \texttt{set\_exposure\_time} -- used to modify the exposure time
  \item \texttt{set\_camera\_gain} -- used to modify the camera gain
  \item \texttt{set\_avg\_diffs} -- used to modify the number of differences used to determine the threshold
  \item \texttt{set\_threshold\_mult} -- used to modify the threshold multiplier
  \item \texttt{set\_frames\_to\_acquire} -- used to modify the max. amount of frames to acquire
\end{enumerate}
