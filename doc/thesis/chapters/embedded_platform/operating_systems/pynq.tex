\subsection{PYNQ}
\label{subsec:embedded_platform:operating_systems:pynq}

PYNQ is an open-source project from Xilinx.
The name is a combination of Python and Zynq, the xilinx processor family.
The main goal of PYNQ is to make it easier for developers to design applications for the Zynq devices. 
Specially, PYNQ works with Overlays, which allow using programmable logic without knowledge of programming in vhdl/verilog.
Avnet distributes a prebuilt image for the Ultra96v2 board.
% cite: https://pynq.readthedocs.io/en/v2.5.1/

\paragraph{Build PYNQ}
Avnet provides source files and instructions for building PYNQ to run on the Ultra96v2 board.
First, the \texttt{Ultra96-PYNQ} repository are cloned and checked out on the prefered version.
PYNQ version 2.5 needs Xilinx tools 2019.1.
The \texttt{PYNQ} repository is needed as well.
The setup\_host.sh script in the repository allows to install all required tools on the host.
As a next step the operating system can be built by the make command and the directory as a parameter.
It is possible to add recipes in the folder \texttt{Ultra96/petalinux\_bsp\_v2/meta-user} to install user specific applications. 
% cite: https://github.com/Avnet/Ultra96-PYNQ

\paragraph{Setup PYNQ}
The wifi configuration scripts (\texttt{interfaces} and \texttt{wpa\_supplicant.conf}) are copied to the SD card to communicate via \acrfull{ssh} after first boot.
To work without password exchanges on the board, the public key for both the Root and the Xilinx user is exchanged.
In a next step, some required packages are installed such as matplotlib or PysimpleGUI.
After this, PYNQ is patched to be compatible with Vitis AI.
This is done by the \texttt{DPU-PYNQ} repository from Xilinx.
This repository contains a makefile which updates pynq, installs \acrfull{xrt} at version 2019.2 and a small Python script to control the DPU clocks.