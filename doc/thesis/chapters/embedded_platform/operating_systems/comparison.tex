\subsection{Comparison}
\label{subsec:embedded_platform:operating_systems:comparison}
Table \ref{tab:compare_os} shows a quick overview of the main functions and properties of the two \acrlong{os}

\begin{table}[hb]
  \caption{Compare PYNQ and Petalinux}
  \label{tab:compare_os}
  \centering
  \begin{tabular}{lll}
    \toprule
    \textbf{} & \textbf{PYNQ} & \textbf{Petalinux} \\
    \midrule
    Kernelversion & 4.19.0 & 4.19.0 \\
    Basic distribution & Ubuntu \cite{pynq_presentation} & none \\
    Supporting Baumer Camera & (yes) \cite{baumer_prog_guide} & no \\
    Wifi function & yes & (yes) \\
    Package manager & \acrshort{apt} & none, installable \\
    \acrshort{dpu} initialization & with overlays at runtime \cite{pynq_overlays} & while booting \cite{petalinux_user_guide} \\
    \bottomrule
  \end{tabular}
\end{table}

PYNQ is based on Ubuntu.
Because Baumer supports Ubuntu, the Baumer camera should work with this \acrlong{os}.
However, installing the official .deb package does not work.
The workaround is to use the installing package for a Rock64 board which Baumer distributed for a while.
A Rock64 development board consists of a Quad-Core ARM Cortex A53 processor just as the Ultrascale+ family.
Further mysqlclient and pq5 has to be installed to use the Baumer camera.

Although, both \acrshort{os} officially support wifi Petalinux has connecting problems due to a faulty root filesystem.

On PYNQ the \acrfull{apt} is installed.
This is the default package manager on most Debian distributions.
On Petalinux a recipe can be added as described in section \ref{subsec:embedded_platform:operating_systems:petalinux} to install and use a package manager.

