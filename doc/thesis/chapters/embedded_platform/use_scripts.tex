\section{Using the Scripts}
\label{sec:embedded_platform:using_scripts}

The scripts provided in the \texttt{AIonFPGA} repository on GitHub allow building the entire chapter \ref{ch:embedded_platform} with four \texttt{make} commands, once the environment is set up.

\paragraph{Get PYNQ}
The first script in the \texttt{mpsoc/build-pynq/} directory requires internet access, root privileges, the local Wi-Fi configurations and three Debian packages in the \texttt{packages/} folder.
It is important to have the correct three package versions:
\begin{itemize}
  \item \texttt{baumer-gapi-sdk-linux-v2.9.2.22969-Ubuntu-18.04-rock64.deb}
  \item \texttt{libmysqlclient20\_5.7.23-0ubuntu0.18.04.1\_arm64.deb}
  \item \texttt{libpq5\_10.4-0ubuntu0.18.04\_arm64.deb}
\end{itemize}

To prevent the destruction of the own hard disk, specify the correct block device of the SD card.
To get the device, use the \texttt{lsblk} command before and after inserting the SD card.
The new node must be defined for example as \texttt{DISK=/dev/sdb} in the Makefile.

The Wi-Fi network name and password must be set in the \texttt{config/wpa\_supplicant.conf} file.

The whole command takes about two hours if all downloads have to be done.

\paragraph{Build DPU}
Building the \acrshort{dpu} with the \texttt{make} command requires internet access to download the \texttt{DPU-PYNQ} repository.
Furthermore, the Xilinx tools, \acrshort{xrt} and Vitis must be installed (version 2019.2).
If Vitis is not installed in the \texttt{/opt/xilinx/2019.2/} directory and \acrshort{xrt} in the \texttt{/opt/xilinx/} directory, they must be redefined in the Makefile.
The building speed depends on the architecture of the \acrshort{dpu} and the host computer.
It takes about \SI{25}{min} for the smallest architecture (B512) and \SI{50}{min} for a B2304 architecture with one \acrshort{dpu}.
To remove the \acrshort{dpu}, the \texttt{make clean-dpu} target is defined in the Makefile.
Running \texttt{make clean} will remove both the \acrshort{dpu} and the downloaded \texttt{DPU-PYNQ} repository.

\paragraph{Deploy Model}
Deploying the model can be done by using the Makefile in the \texttt{mpsoc/cnn-model/} directory.
The script starts a Docker container and compiles the application.
To run the make command, Docker must be installed on version 19.03.1 or higher \cite{vitis_ai_user_guide}.
When quantizing and compiling for the first time, the image with a size of \SI{3.5}{GB} is downloaded from Docker Hub.
The \texttt{build-dpu} Makefile must be executed before the model is deployed.

\paragraph{Set up PYNQ}
After executing the other three Makefiles, insert the SD card and power the Ultra96-V2 board.
The Makefile in the \texttt{mpsoc/pynq-setup/} directory can now be executed.

To execute the script successfully, the IP address of the board must be changed to the respective IP address.
This address can be read from the Wi-Fi access point.

In a first step the \acrshort{ssh} keys are exchanged by the \texttt{make} command.
Therefore, the host computer needs a key pair in the default directory (\texttt{/home/<user>/.ssh} where \texttt{<user>} is replaced with the respective username).

Running the script takes about four hours and the passwords for the user and the root user of the board are needed at the beginning.
Both passwords are \textit{xilinx} by default.
