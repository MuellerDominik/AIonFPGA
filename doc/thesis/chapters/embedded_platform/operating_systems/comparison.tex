\subsection{Comparison}
\label{subsec:embedded_platform:operating_systems:comparison}

Table \ref{tab:compare_os} provides an overview of the main functions and properties of the two \acrlongpl{os}.

\begin{table}
  \caption{Comparison of PYNQ and PetaLinux}
  \label{tab:compare_os}
  \centering
  \begin{tabular}{lll}
    \toprule
     & \textbf{PYNQ} & \textbf{PetaLinux} \\
    \midrule
    Kernel version & 4.19.0 & 4.19.0 \\
    Basic distribution & Ubuntu \cite{pynq_presentation} & - \\
    Baumer camera support & (yes) \cite{baumer_prog_guide} & no \\
    Wi-Fi functionality & yes & (yes) \\
    Package manager & \acrshort{apt} & none, installable \\
    \acrshort{dpu} initialization & with overlays at runtime \cite{pynq_overlays} & while booting \cite{petalinux_user_guide} \\
    \bottomrule
  \end{tabular}
\end{table}

PYNQ is based on Ubuntu.
Since Baumer supports Ubuntu, the Baumer camera should work with this \acrlong{os}.
However, the installation of the official \texttt{.deb} package does not work.
The workaround is to use the installation package for the Rock64 board that Baumer distributed for a while in the past.
The Rock64 development board features a quad-core ARM Cortex A53 processor like the UltraScale+ family.
Additionally, \texttt{mysqlclient} and \texttt{pq5} must be installed to use the Baumer camera.

Although both \acrlongpl{os} officially support Wi-Fi, PetaLinux has connecting problems due to a faulty root filesystem.

On PYNQ the \acrfull{apt} is installed.
This is the default package manager on most Debian-based distributions.
On PetaLinux, a recipe can be added to install and use a package manager, as described in section \ref{subsec:embedded_platform:operating_systems:petalinux}.
