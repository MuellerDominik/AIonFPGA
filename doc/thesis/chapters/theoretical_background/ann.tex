\section{Artificial Neural Network}
\label{sec:theoretical_background:ann}

% An \acrlong{ann} is a model used in \acrshort{ml}, % p.~727--
An \acrlong{ann} is a model used in \acrshort{ml} which consists of artificial neurons and weighted connections between those neurons.
% It usually consists of an input layer, several hidden layers and an output layer.
They usually features an input layer, several hidden layers and an output layer.
The individual layers consist of an arbitrary number of artificial neurons, which are connected to one another \cite[p.~33-36]{nn}.

% The simplest \acrshort{ann} is the feedforward neural network, where the information flows in only one direction.
The simplest \acrshort{ann} is the feedforward neural network, where the information propagates in only one direction.
% artificial neurons
% sums up inputs + bias
% applies activation function
% A common propagation function is the weighted sum of the neurons from the previous layer.
A common propagation function is the weighted sum of the neurons from the previous layer.
% The output of each neuron is multiplied with the weight ascociated to the connection between the two neurons.
Therefore, the output of each neuron $i$ is multiplied with the weight $w_{i, j}$ ascociated to the connection between the two neurons.
% These values are then summed up and a bias term is added.
% These values are then summed up and added to a bias term.
These values are summed up and added to a bias term.
In a last step, an activation function is applied, which results in equation \ref{eq:artificial_neuron}.

\begin{equation}
  % y_j = a\left(\sum\limits_{i=1}^{m} (x_i \cdot w_{i, j}) + b\right)
  % y_j = a\left(\sum\limits_{i=1}^{m} x_i w_{i, j} + b\right)
  y_j = a\left(\sum\limits_{i} x_i \cdot w_{i, j} + b\right)
  \label{eq:artificial_neuron}
\end{equation}

where

\begin{tabular}{lll}
  $y_j$ & = & output of neuron $j$ \\
  $a(.)$ & = & activation function \\
  % $m$ & = & number of neurons in the previous layer \\
  $x_i$ & = & output of neuron $i$ from the previous layer \\
  $w_{i, j}$ & = & weight $i, j$ connecting the output of neuron $i$ with the input of neuron $j$ \\
  $b$ & = & bias \\
\end{tabular}
\\




\subsection{Activation Function}
\label{subsec:theoretical_background:ann:activation_function}

An activation function is responsible for transforming the input of an artificial neuron to its output.
% Activation functions are used to introduce a nonlinearity.
% They are usually used to introduce a nonlinearity to to solve problems which are not trivial.
There exists a variety of different linear and non-linear activation functions.
However, non-linear activation functions are preffered due to their ability to learn more complex structures in the data.
The sigmoid and hyperbolic tangent activation functions were traditionally very popular but they both suffer from a saturation problem.
% Most commonly, the \acrfull{relu} activation function is used, which is shown in equation \ref{eq:relu}.
% At the moment, the \acrfull{relu} activation function is  the most commonly used activation function is the \acrfull{relu} activation function
% The \acrfull{relu} activation function is currently the most commonly used.
Currently, the \acrfull{relu} is the most commonly used activation function in the field of deep learning.
It is a piecewise-defined function, which outputs the input if it positive and zero otherwise.
This is shown in equation \ref{eq:relu}.
Furthermore, the \acrshort{relu} activation function requires only a single comparison and is therefore not very computationally expensive \cite{relu}.

\begin{equation}
  a(x) = x^+ = \max(0, x)
  \label{eq:relu}
\end{equation}

% relu (incl. plot)
% maybe sigmoid or tanh
