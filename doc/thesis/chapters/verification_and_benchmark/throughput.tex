\section{Throughput}
\label{sec:verification_and_benchmark:throughput}

All steps which are passed through in the while true loop are measured. 
These are:
\begin{itemize}
	\item get raw Bayer from \acrshort{ram}
	\item transform Bayer to \acrshort{bgr} 
	\item resize frame size %from \SI{1280}{px} $\times$ \SI{1024}{px} to \SI{320}{px} $\times$ \SI{256}{px}
	\item normalize the image which is a float32 division
	\item run inference on the \acrshort{dpu}
	\item run softmax function
	\item add weighting
\end{itemize}

To measure the time needed for each step, the python module datetime was used.
The function \texttt{datetime.now()} returns the current local date and time.

To measure these steps the main python file is slightly modified.
The initialization is not changed at all.
The infinite loop is removed, so that only one throw is captured and the application ends afterwards.
Furthermore, the start and end time are recorded around each of the above steps.

The times are stored in a three-dimensional dictionary.
In the first dimension is the measured command.
The second dimension indicates if it was the start or end of the command.
Since the times vary from frame to frame, several images are taken and averaged in a final step.
The times are appended at the end of the corresponding list.
Listing \ref{lst:measure_time} shows an example for the color transformation.

\begin{lstlisting}[style=python, caption={Measure needed time for the color transformation}, label=lst:measure_time]
  meas_time['Transform color']['start'].append(datetime.now())
  frames[idx] = cv2.cvtColor(raw_frame, cv2.COLOR_BayerBG2BGR)
  meas_time['Transform color']['end'].append(datetime.now())
\end{lstlisting}

The time which is needed for each step is the difference between end and start.
When all images have been computed, the average value is calculated for each step.
This is printed on the console and written to a file.

Table \ref{tab:measured_times} shows the measured times.

\begin{table}[hb]
	\caption{Measured time for the different steps}
	\label{tab:measured_times}
	\centering
	\begin{tabular}{ll}
		\toprule
		\textbf{Step} & \textbf{Time} \\
		\midrule
		get raw Bayer & \SI{4}{ms} \\
		transform & \SI{4}{ms} \\
		resize & \SI{4}{ms} \\
		normalize & \SI{4}{ms} \\
		run inference & \SI{4}{ms} \\
		run softmax & \SI{4}{ms} \\
		add weighting & \SI{4}{ms} \\
		\midrule
		Total time & \SI{20}{ms} \\
		\bottomrule
	\end{tabular}
\end{table}

By calculating the reciprocal value of the total time required, the frame rate can be determined:
\begin{equation}
  \text{framerate} = \frac{1}{T_{tot}} = \frac{1}{\SI{20}{ms}} = \SI{50}{fps}
  \label{eq:achieved_framerate}
\end{equation}
\todo[inline]{edit numbers}
