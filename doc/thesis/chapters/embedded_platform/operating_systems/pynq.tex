\subsection{PYNQ}
\label{subsec:embedded_platform:operating_systems:pynq}

PYNQ is an open-source project from Xilinx.
The name is a combination of Python and Zynq, the Xilinx processor family.
The main goal of PYNQ is to make it easier for developers to design applications for the Zynq devices. 
In particular, PYNQ works with overlays that allow the use of \acrlong{pl} without programming knowledge in VHDL or Verilog.
Avnet distributes a pre-built image for the Ultra96-V2 board \cite{pynq_intro}.

\paragraph{Build PYNQ}
Avnet provides source files and instructions for building PYNQ to run on the Ultra96-V2 board.
First, the \texttt{Ultra96-PYNQ} repository is cloned and checked out on the preferred version.
PYNQ version 2.5 requires Xilinx tools 2019.1.
The \texttt{PYNQ} repository is also required.
The \texttt{setup\_host.sh} script in the repository allows the installation of all necessary tools on the host.
% The next step the operating system can be built by the make command and the directory as a parameter.
The next step is to build the operating system with the \texttt{make} command and the directory as a parameter.
It is possible to add recipes in the folder \texttt{Ultra96/petalinux\_bsp\_v2/meta-user} to install user specific applications \cite{avnet_pynq_github}. 

\paragraph{Setup PYNQ}
The Wi-Fi configuration scripts (\texttt{interfaces} and \texttt{wpa\_supplicant.conf}) are copied to the SD card to communicate via \acrfull{ssh} after first boot.
% To work without password exchanges on the board, the public key for both the Root and the Xilinx user are exchanged.
To work on the board without constant password exchanges, the public key is exchanged for both the root and the Xilinx user.
In a next step, some required packages are installed such as \texttt{matplotlib} or \texttt{PysimpleGUI}.
% After this, PYNQ is patched to be compatible with Vitis AI.
PYNQ is then patched to be compatible with Vitis AI.
This is done using the \texttt{DPU-PYNQ} repository from Xilinx.
This repository contains a Makefile that updates PYNQ, installs the \acrfull{xrt} version 2019.2 and a small Python script to control the \acrfull{dpu} clocks.
The patch also updates \acrshort{opencv} to version 3.4.3.
Next, the repository from this project is cloned to the board.
This way all created programs and packages are already at the expected location.
Finally, the \acrshort{dpu} model and the required files for programming the \acrfull{pl} are copied to the target over \acrfull{scp}.

To always run the \acrshort{ai} application at startup, a systemd service is used.
This service can be started manually with the following command:
\begin{lstlisting}[style=bash, caption={}, label=lst:start_service]
  systemctl start aionfpga.service
\end{lstlisting}
A systemd service requires at least a description in the \texttt{[Unit]} section, a type and the command to execute in the \texttt{[Service]} section and other configurations in the \texttt{[Install]} section \cite{systemd}.

% To be able to run a graphical application, the executed command is a shell script.
To run a graphical application, the executed command starts a shell script.
The shell script exports the display and then starts the \acrshort{ai} application.

For the service to be executed at startup, it must be activated with the following command:
\begin{lstlisting}[style=bash, caption={}, label=lst:enable_service]
  systemctl enable aionfpga.service
\end{lstlisting}
