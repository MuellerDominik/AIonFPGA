\chapter*{Introduction}
\addcontentsline{toc}{chapter}{Introduction}
\label{ch:introduction}

Humanity tries to understand how thinking works since thousands of years and is making considerable progress.
One of the newest fields in science and engineering is the \acrfull{ai} which goes even further than just understanding the thinking process.
The attempt is to build own intelligent entities \cite{ai}.
For example, in most vehicles today, the human being chooses which reaction is the best.
However, humans are influenced, distracted and have reactions that are simply too slow to avoid accidents.
When these tasks are given to a computer, something in the background is needed that is able to recognize dangers and react appropriately.
To do this, an image must be taken and evaluated.
Because the dangers are not always the same, intelligence is needed to make the decisions.
This is where \acrshort{ai} comes in.
The progress in the field of self-driving cars through \acrshort{ai} has been considerable in recent years.
All Tesla cars being produced now have full self-driving hardware and only need a software update to replace the driver \cite{tesla_self_driving_cars}.
Another example is the optical quality control.
Thanks to computers, the accuracy can be optimized and the throughput can be increased compared to human workers.
Both examples require fast and accurate image recognition which is usually done by a so called \acrfull{cnn}. 

In order to keep up with the current state of development, the \acrfull{ise} which is part of the \acrfull{fhnw} has started this project that deals with these two aspects.
There are two main targets in the project:

\begin{enumerate}
	\item Showing the \acrshort{fhnw} how to run hardware accelerated \acrshort{ai}.
	\item Building an exhibition object for the \acrshort{ise}, which can be presented at fairs. 
\end{enumerate}

Apart from the hardware, which is an Ultra96-V2 development board, the procedure and the order was not defined more precisely, which allowed some leeway.
The task was defined more precisely by the students as follows:

To meet the requirements of fast image recognition, objects are thrown past a high-speed camera.
These objects are taken from a set of 22 items such as a floorball or a stuffed bunny.
The \acrshort{cnn} shall be described and trained with an end-to-end open source platform for machine learning called Tensorflow \cite{tensorflow_main}.
The throwing booth should function as a plug and play device.
This makes it easier to use at a fair.
In appendix \ref{app:problem_statement} the task definition can be seen.

In a previous project work the throwing booth was constructed and built up.
A camera and the lighting were evaluated, bought and mounted.
Furthermore a data set was collected, which contains at least 480 pictures of each object. 
For each image in the database, labels have been set to indicate what kind of object it is.
Also it is noted if the picture is not good (for example because a hand was on the picture) or if the object is cut off at the edge  of the image.

In this project an \acrfull{os} for the processor of the Ultra96-V2 was set up.
The \acrshort{os} can control the \acrfull{fpga}, which is included in the \acrfull{mpsoc}.
A \acrfull{dpu} was programmed on this \acrshort{fpga}, which performs the high speed image recognition.
The trained neural network was quantized before.
This makes it possible to execute the image recognition by using C-code commands.
The results were verified on accuracy and throughput.

This report contains five main parts.
The first one is the theoretical background about \acrfull{ai}.
The second part shows how the \acrlong{cnn} is trained.
Next comes a chapter that describes the throwing booth.
The chapter embedded platform introduces two operating systems and describes how to use them.
The results of the \acrshort{cnn} are documented in the chapter verification and benchmark.
After these chapters there is a conclusion and the personal experiences with Xilinx and Avnet.

All components developed for this project fall under the Apache 2.0 license and are freely available via GitHub.