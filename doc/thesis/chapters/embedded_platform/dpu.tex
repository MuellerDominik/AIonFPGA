\section{DPU}
\label{sec:embedded_platform:dpu}

The Xilinx \acrfull{dpu} is developed to run convolutional neural networks on an \acrshort{fpga} of the Zynq UltraScale+ or Zynq-7000 family.
\acrshort{dpu}v1 and \acrshort{dpu}v3 are for cloud computing on Alveo cards and \acrshort{dpu}v2 is for Zynq devices. 
% Due to parallelization of the calculations in a neural network, the inference can be accelerated considerably.
Due to the parallelization of the calculations in a neural network, the inference can be accelerated considerably.
The \acrshort{dpu} is configured by an AXI slave interface and accesses instructions by an AXI master interface.
% It can be configured to make best use of the size of the \acrshort{fpga}.
It can be configured so that the size of the \acrshort{fpga} is used optimally.
The encrypted \acrshort{rtl} design files are available on GitHub.
To use the \acrshort{dpu}, a device driver is required \cite{dpu_product_guide}.
This driver is included in the Xilinx Vitis AI development kit \cite{dpu_product_guide_v3_2}.

\subsection{Configuration}
\label{subsec:embedded_platform:dpu:configuration}
\paragraph{Number of DPUs}
The number of \acrshortpl{dpu} is set in the \texttt{prj\_config} configuration file with \texttt{nk=dpu\_xrt\_top:x}.
The variable \texttt{x} stands for the number of \acrshortpl{dpu}.
Depending on the UltraScale+ device, one to three \acrshort{dpu} cores can be implemented.
For example, only one core is supported on ZU2 to ZU5 devices.
On a ZU7 device it is possible to implement two cores, while on a ZU9 device even three cores can be implemented.

\paragraph{Clock}
The clock frequency is also defined in the \texttt{prj\_config} file.
For the ZU3 device, the maximum clock frequency is \SI{370}{MHz}.
Due to a hardware design flaw on the Ultra96-V2 board, the maximum frequency is \SI{300}{MHz} with PYNQ and \SI{150}{MHz} with PetaLinux.
At higher frequencies the \acrfull{pmic} reboots the entire system.
uld be possible to fix this problem, but it requires both hardware and firmware changes \cite{pmic_issue}.
The frequency can be adjusted by setting the variable \texttt{freqHz=150000000:DPUCZDX8G\_1.aclk}.
% Another way is using the predefined frequencies from Xilinx.
Another possibility is to use the predefined frequencies from Xilinx.
Table \ref{tab:frequencies_ids} lists the possible values.

\begin{table}
  \caption{Ids and their Frequencies}
  \label{tab:frequencies_ids}
  \centering
  \begin{tabular}{ll}
    \toprule
    \textbf{ID} & \textbf{Frequency} \\
    \midrule
    0 & \SI{150}{MHz} \\
    1 & \SI{300}{MHz} \\
    2 & \SI{75}{MHz} \\
    3 & \SI{100}{MHz} \\
    4 & \SI{200}{MHz} \\
    5 & \SI{400}{MHz} \\
    6 & \SI{600}{MHz} \\
    \bottomrule
  \end{tabular}
\end{table}

% To use the ID, replace freqHz with id and 15000000 with 0.
To use the ID, \texttt{freqHz} must be replaced by \texttt{id} and \num{15000000} by 0.
% Besides this clock the \texttt{ap\_clk\_2} must be set as well.
In addition to this clock, \texttt{ap\_clk\_2} must also be set.
This clock should be twice the frequency of \texttt{aclk} and is the one that should not exceed \SI{300}{MHz}.

\paragraph{DPU Architecture}
The following configurations are set in the \texttt{dpu\_conf.vh} file.
The \acrshort{dpu} IP can be configured with different convolution architectures.
They are related to the parallelism of the convolution unit.
On a ZU3EG device the best results can be achieved with a B2304 architecture.
\num{2304} refers to the maximum operations per clock.
Therefore, a higher number corresponds to a higher throughput.
% B2304 architecture is used in this project.
In this project the B2304 architecture is used.
Table \ref{tab:arch_parallelism} lists all available architectures and the correlation to the parallelism.

\begin{table}
  \caption{Parallelism for Different Convolution Architectures \cite{dpu_product_guide}}
  \label{tab:arch_parallelism}
  \centering
  \begin{tabular}{lllll}
    \toprule
     & \textbf{Pixel} & \textbf{Input Channel} & \textbf{Output Channel} & \textbf{Peak Ops} \\
    \textbf{Architecture} & \textbf{Parallelism} & \textbf{Parallelism} & \textbf{Parallelism} & \textbf{(ops per clock)} \\
    \midrule
    B512 & 4 & 8 & 8 & 512 \\
    B800 & 4 & 10 & 10 & 800 \\
    B1024 & 8 & 8 & 8 & 1024 \\
    B1152 & 4 & 12 & 12 & 1150 \\
    B1600 & 8 & 10 & 10 & 1600 \\
    B2304 & 8 & 12 & 12 & 2304 \\
    B3136 & 8 & 14 & 14 & 3136 \\
    B4096 & 8 & 16 & 16 & 4096 \\
    \bottomrule
  \end{tabular}
\end{table}

\paragraph{RAM Usage}
The RAM usage can be set to high or low.
If the chip has no UltraRAM, the on-chip \acrshort{bram} is used and the UltraRAM must be disabled.
One block has a size of \SI{36}{kB} and for the B2304 architecture 167 such blocks are used for \acrshort{ram} low and 211 for \acrshort{ram} high.
If UltraRAM is available, several other defines need to be specified.
Listing \ref{lst:config_dpu} shows an example of how to do this with a B4096 architecture.
The UltraRAM numbers can be found in the Zynq \acrshort{dpu} \acrshort{ip} product guide.
The ZU3EG device has 216 \acrfull{bram} blocks and no UltraRAM.
% Because the RAM for the \acrshort{pl} is not used for any other calculation, RAM high can be activated and UltraRAM should be disabled.
Since RAM for the \acrshort{pl} is not used for any other calculation, RAM high can be enabled and UltraRAM must be disabled.
Practical tests showed throughput time reductions of about \SI{20}{\percent} with \acrshort{ram} high compared to \acrshort{ram} low.

\paragraph{Channel Augmentation}
This feature improves the efficiency of the \acrshort{dpu} when the number of input channels is lower than the available channel parallelism.
Activating this function costs additional logic resources.
% In this CNN there are three input channels and therefore, channel augmentation is enabled.
This \acrshort{cnn} has three input channels, and therefore, channel augmentation is enabled.

\paragraph{Depthwise Convolution}
In standard convolution, operations are performed separately for each input channel with a specific kernel and the results are combined across all channels.
In depthwise separable convolution, the operations are performed in two steps: depthwise convolution and pointwise convolution.
Depthwise convolution is enabled because it allows for more flexibility in the \acrshort{cnn} architecture.

\paragraph{AveragePool}
% Average pooling can be done on the \acrshort{dpu} or not.
Average pooling can be enabled or disabled on the \acrshort{dpu}.
The supported size ranges from $2\times2$ to $8\times8$, with only square sizes supported.
Average pooling is activated to accelerate the whole process.

\paragraph{ReLU Type}
By default the \acrshort{relu} and \acrshort{relu}6 activation functions are supported.
It is possible to include the Leaky\acrshort{relu} function as well with low resource consumption.
% Because of the low resource consumption Leaky\acrshort{relu} is also included.
Due to the low consumption of resources, Leaky\acrshort{relu} is also included.
So if the \acrshort{cnn} is changed, the \acrshort{dpu} does not have to be rebuilt.

\paragraph{Softmax}
The softmax function can be implemented in both hardware and software.
% Implemented on hardware, the function is about 160 times faster than a software implementation.
The function implemented on hardware is about 160 times faster than a software implementation.
The function needs approximately \SI{10000}{\acrshortpl{lut}}, four \acrshort{bram} blocks and 14 \acrshort{dsp} slices.
If the resources run out, the \acrlong{dnnc} implements the function independently in software.
The softmax function is not available on Zynq-7000 devices \cite{dpu_product_guide}. 

% Implementing the softmax in hardware is not done using a define, but during the build process.
The implementation of the softmax function in hardware is not done via a define, but during the build process.
The \texttt{make} command is \texttt{make KERNEL=DPU\_SM} instead of \texttt{make KERNEL=DPU}.
The softmax layer is not implemented in hardware in this project.

\paragraph{Configuration File}
With all configuration settings, the \texttt{dpu\_config.vh} file for a ZU3EG device could look like this:

\begin{lstlisting}[style=bash, caption={DPU configuration}, label=lst:config_dpu]
  `define B2304
  `define URAM_DISABLE 

  // Config URAM
  `ifdef URAM_ENABLE
    `define def_UBANK_IMG_N          5
    `define def_UBANK_WGT_N          17
    `define def_UBANK_BIAS           1
  `elsif URAM_DISABLE
    `define def_UBANK_IMG_N          0
    `define def_UBANK_WGT_N          0
    `define def_UBANK_BIAS           0
  `endif
  `define RAM_USAGE_HIGH
  `define CHANNEL_AUGMENTATION_ENABLE
  `define DWCV_ENABLE
  `define POOL_AVG_ENABLE
  `define RELU_LEAKYRELU_RELU6
  `define DSP48_USAGE_LOW
\end{lstlisting}

\subsection{Building}
\label{subsec:embedded_platform:dpu:building}

The \texttt{DPU-PYNQ} repository from Xilinx, which is used to patch PYNQ, can be used to build the \acrshort{dpu}.
It needs to be cloned to the host computer.
By cloning the repository the \texttt{Vitis-AI} repository is cloned as a submodule.
It contains the \texttt{DPU-TRD/} directory to build a \acrshort{dpu}.
In a next step the created configuration files must be copied to the \texttt{boards/Ultra96/} directory.
Now the Xilinx tools Vitis and \acrshort{xrt} must be sourced as shown in listing \ref{lst:source_tools}.
The Makefile checks if all required tools are installed and sourced and then starts Vivado.
During the building process, which can take more than an hour, the bitstream is generated. 
After executing the Makefile in the \texttt{boards/} directory with the parameter \texttt{BOARD=Ultra96}, the files \texttt{dpu.bit}, \texttt{dpu.hwh} and \texttt{dpu.xclbin} are created and located in the same directory where the configuration files were stored.
These files can be copied to the board and the customized \acrshort{dpu} is ready for use.

When working with PetaLinux the generated \texttt{BOOT.BIN} file is also of interest.
This file is located in the \texttt{boards/Ultra96/binary\_container\_1/sd\_card/} directory.
The \texttt{BOOT.BIN} file is responsible for loading the \acrshort{dpu} during boot with the \texttt{.xclbin} file.
The \texttt{.xclbin} file file should be located under \texttt{/usr/lib/} on the board.
Using PYNQ, the \acrshort{fpga} can be programmed when the application is started.
