\section{Using the Scripts}
\label{sec:embedded_platform:using_scripts}

The scripts provided in the \texttt{AIonFPGA} repository on GitHub allow it to build the whole chapter \ref{ch:embedded_platform} with four make commands once the environment is set up.

\paragraph{Get PYNQ}
The first script in the \texttt{mpsoc/build-pynq} needs internet access, root privileges, the local Wi-Fi configurations and three Debian packages in the \texttt{packages} folder.
It is important to have the correct three package versions:
\begin{itemize}
	 \item \texttt{baumer-gapi-sdk-linux-v2.9.2.22969-Ubuntu-18.04-rock64.deb}
	 \item \texttt{libmysqlclient20\_5.7.23-0ubuntu0.18.04.1\_arm64.deb}
	 \item \texttt{libpq5\_10.4-0ubuntu0.18.04\_arm64.deb}
\end{itemize}

It is important to specify the correct block device for the SD card.
To get the device, use the \texttt{lsblk} command before and after inserting the SD card.
The new node needs to be defined as for example \texttt{DISK = /dev/sdb} in the makefile.

The Wi-Fi net name and password must be set in the \texttt{config/wpa\_supplicant.conf} file.

The whole command takes about two hours if all downloads need to be done.

\paragraph{Build \acrshort{dpu}}
Building the \acrshort{dpu} with the make command needs internet access to download the \texttt{DPU-PYNQ} repository.
Furthermore, the Xilinx tools \acrshort{xrt} and Vitis need to be installed on version 2019.2.
When Vitis is not installed in \texttt{/opt/xilinx/2019.2} and \acrshort{xrt} in the \texttt{/opt/xilinx} directory they need to be redefined in the makefile.
The building speed depends on the architecture of the \acrshort{dpu} and the host computer.
It takes about \SI{25}{min} for the smallest architecture (B512) and \SI{50}{min} for a B2304 architecture with one \acrshort{dpu}.
To remove the \acrshort{dpu} the \texttt{make clean-dpu} is defined in the makefile.
Running the \texttt{make clean} removes the \acrshort{dpu} as well as the downloaded \texttt{DPU-PYNQ} repository.

\paragraph{Deploy Model}
Deploying the model can be done by the makefile in the \texttt{mpsoc/cnn-model} directory.
The script starts a docker container and compiles the application.
To run the make command docker must be installed on version 19.03.1 or above \cite{vitis_ai_user_guide}.
When quantizing and compiling for the first time the image with a size of \SI{3.5}{GB} is downloaded from Docker Hub.
The build-dpu makefile must already be executed when deploying the model.

\paragraph{Set up PYNQ}
After executing the other three makefiles, insert the SD card and power the Ultra96-V2 board.
The makefile in the \texttt{mpsoc/pynq-setup} directory can now be executed.

To successfully run the script the IP address of the board needs to be edited to the received IP address.
This address is viewable at the access point.

In a first step the \acrshort{ssh} keys are exchanged by the make command.
Therefore the host computer needs a key pair at the default directory (/home/\textit{user}/.ssh where \textit{user} is substituted with the current user name).

Running the script takes about four hours and the passwords for the user and the root user for the board are needed at the beginning.
Both passwords are just xilinx by default.
