\section{Machine Learning}
\label{sec:theoretical_background:ml}

\Acrlong{ml} is a branch of \acrlong{ai} that essentially aims to give computers the ability to learn from data.
Therefore, it deals with the study of algorithms that computers use to perform certain tasks without using explicit instructions \cite[p.~97]{deeplearningbook}.

% ------------------------
\paragraph{Generalization}
The goal of the learning process is to generalize and not to memorize the samples of the training dataset.
A model that fails to generalize can be subject to underfitting or overfitting.
On the one hand, underfitting occurs when the model is not able to reach an acceptable classification accuracy.
Overfitting, on the other hand, refers to the situation where the gap between the training accuracy and the validation accuracy is too large \cite[p.~108--114]{deeplearningbook}.

% ----------------------------------------------
\paragraph{Supervised and Unsupervised Learning}
\Acrlong{ml} algorithms can generally be divided into supervised and unsupervised learning algorithms.
The distinction is based on the type of datasets used during training \cite[p.~102--105]{deeplearningbook}.

Supervised learning algorithms use labeled datasets to learn.
The algorithm takes a guess on a piece of data from the dataset and this guess is then checked against the label (i.e. the correct answer).
This enables the algorithm to adjust itself and therefore to improve future predictions.

Unsupervised learning algorithms, however, work with unlabeled data and therefore only operate on the input data.
The algorithm must learn to make sense of the input data, as it has no means of correcting itself.
Unsupervised learning is usually used in problems that involve finding groups in data or summarizing the distribution of data.

% -----------------------
\paragraph{Deep Learning}
Deep learning is a subset of \acrlong{ml} that was developed to achieve better results than traditional algorithms.
Deep learning algorithms are combinations of a dataset, a cost function, an optimizer and a model.
The best results are achieved by using deep learning architectures in combination with large datasets.
The word \textit{deep} in \textit{deep learning} refers to the large number of layers in a network \cite[p.~151--152]{deeplearningbook}.
