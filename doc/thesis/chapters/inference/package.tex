\section{Package}
\label{sec:inference:package}
% \todo[inline]{change paragraphs to sections?, todos, cleanup}

% The inference application as well as the Python scripts used during the training require a lot of similar functionality (e.g. database connectivity, settings).
Both the inference application and the Python scripts used during the training require a lot of similar functionality (e.g. settings, constants, database connectivity).
For this reason the \texttt{fhnwtoys} Python package, consisting of several modules (\texttt{.py} files), was created.

\subsection{Structure}
\label{subsec:inference:package:structure}
There are three main modules:
\begin{enumerate}
  \item Definitions (\texttt{definitions.py})
  \item Settings (\texttt{settings.py})
  \item Enums (\texttt{enums.py})
\end{enumerate}

The \texttt{fhnwtoys.definitions} module contains, as the name suggests, several definitions (functions and classes).
These definitions are only used during the training and verification process on the host.

The \texttt{fhnwtoys.settings} module contains various names and settings that are used throughout the project (e.g. the names of the objects, file names, camera settings).

The \texttt{fhnwtoys.enums} module contains a bunch of enumerations which provide a convenient way to work with a set of related constants \cite{}. % todo: cite https://docs.python.org/3/library/enum.html
% These make the code clearer as they 
% This improves the readability of the code by removing the necessity of magic numbers
This improves the readability of the code by removing magic numbers (unexplained numerical constants).

% There are two additional modules, which are meant to be for importing:
% There are two additional modules, which can be used to import the required main modules:
% Additionally, there are two modules which can be used to import the required main modules:
Additionally, there are two modules that can be used to import the required main modules:
\begin{enumerate}
  \item Training (\texttt{training.py})
  \item Inference (\texttt{inference.py})
\end{enumerate}

Importing the \texttt{fhnwtoys.training} module provides access to all three main modules.

Importing the \texttt{fhnwtoys.inference} module is the same as importing the \texttt{fhnwtoys} package directly, as it is an alias for the \texttt{fhnwtoys.\_\_init\_\_} module.
However, explicitly importing the \texttt{fhnwtoys.inference} module is the preffered method.
This provides acces to the \texttt{fhnwtoys.settings} and \texttt{fhnwtoys.enums} modules.

% % -------------------------------
% \paragraph{Importing the Package}
\subsection{Importing the Package}
\label{subsec:inference:package:importing_the_package}
Listing \ref{lst:importing_package} shows how to import the \texttt{fhnwtoys} package.
% The imported package is aliased as \texttt{fh}.
For convenience purposes, the imported package is aliased as \texttt{fh}.

\begin{lstlisting}[style=python, caption={Importing the \texttt{fhnwtoys} Python package}, label=lst:importing_package]
import fhnwtoys.training as fh # training and verification
import fhnwtoys.inference as fh # inference application
\end{lstlisting}

% % ----------------------
% \paragraph{Installation}
\subsection{Installation}
\label{subsec:inference:package:installation}
A setup script is provided which allows for an easy installation of the \texttt{fhnwtoys} package.
% A setup script allows for an easy installation of the \texttt{fhnwtoys} package.
% The \texttt{fhnwtoys} package can be installed from the \texttt{packages/} directory with the following command:
% It can be installed from the \texttt{packages/} directory with the following command:
It can be installed with the following command from the \texttt{packages/} directory:

\begin{lstlisting}[style=bash]
sudo pip3 install -e .
\end{lstlisting}
