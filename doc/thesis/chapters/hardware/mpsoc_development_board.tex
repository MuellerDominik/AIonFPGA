\section{MPSoC Development Board}
\label{sec:hardware:mpsoc_development_board}

The hardware is an Ultra96-V2 board, which is distributed by Avnet.
The \acrfull{mpsoc} placed on it is a Xilinx Zynq UltraScale+ \acrshort{mpsoc} ZU3EG.
This chip is from the Zynq UltraScale+ family and has an UltraScale architecture.
Integrated is a quad-core ARM Cortex A53 processor to run a complete operating system and a dual-core ARM Cortex R5, which makes the Ultra96-V2 hard real-time capable.
The A53 core can be clocked with \SI{1.5}{GHz}, the R5 with \SI{600}{MHz}.

The \acrshort{fpga} on the \acrshort{mpsoc} allows a hardware acceleration of up to a factor of 20 compared to the fastest CPUs \cite{acceleration_xilinx}.
Avnet therefore recommends the board as ideal for the field of high-speed artificial intelligence \cite{ai_resources_xilinx}.
The most important specifications of the \acrshort{mpsoc} are listed in the following table \ref{tab:specs_MPSoC}.

\begin{table}[h]
	\caption{Xilinx Zynq UltraScale+ \acrshort{mpsoc} ZU3EG key features \cite{xilinx_zynq}}
	\label{tab:specs_MPSoC}
	\centering
	\begin{tabular}{ll}
		\toprule
		& \textbf{ZU3EG} \\
		\midrule
		\textbf{Logic Cells} & \SI{154}{k} \\
		\textbf{Flip Flops} & \SI{141}{k} \\
		\textbf{Block RAM} & \SI{7.6}{Mb} \\
		\textbf{DSP Slices} & 360 \\
		\bottomrule
	\end{tabular}
\end{table}

The Ultra96-V2 also features two USB 3.0 ports and \SI{2}{GB} \acrfull{lpddr4} \acrshort{ram}, which are essential for fast image processing.
A \acrfull{mdp} serves as a connection to a monitor \cite{avnet_ultra96v2}.
This guarantees a standalone operation.

