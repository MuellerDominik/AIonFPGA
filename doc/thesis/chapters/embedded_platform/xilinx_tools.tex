\section{Xilinx Tools}
\label{sec:embedded_platform:xilinx_tools}

\paragraph{PetaLinux}
PetaLinux is an embedded Linux \acrfull{sdk}.
It is a development tool that contains everything necessary to build, develop, test, and deploy embedded Linux systems.
In the background PetaLinux is running Yocto functions.
The goal of PetaLinux is to make it easier for developers to build an own embedded Linux \cite{petalinux_user_guide}.

\paragraph{Vivado}
Xilinx offers many tools to work with their hardware.
The traditional way is to design hardware with a \acrfull{hdl} like VHDL or Verilog.
These \acrfull{rtl} descriptions are passed to a synthesis tool and written to the \acrshort{fpga}.
% This functionality is embraced by the Vivado Design Suite HLx Editions.
This functionality is provided by the Vivado Design Suite HLx editions.
The output of Vivado is a bitstream that can be loaded to the target \cite{vivado_user_guide}.

\paragraph{Vitis Unified Software Platform}
% Vitis Unified Software Platform in short sometimes called Vitis provides software developers with the ability to build accelerated applications.
Vitis Unified Software Platform, or Vitis for short, offers software developers the ability to build accelerated applications.
% To communicate with the hardware \acrfullpl{api} are used.
\Acrfullpl{api} are used to communicate with the hardware.
% The Vitis core development kit includes the v++ compiler for the hardware kernel, the g++ compiler for compiling the application to run on an x86 host, and an ARM compiler for cross-compiling the application to run on the embedded processor of a Xilinx device \cite{vitis_user_guide}.
The Vitis core development kit includes the v++ compiler for the hardware kernel and the g++ compiler for compiling the application to run on an x86 host.
It also includes an ARM compiler for cross-compiling the application to run on the embedded processor of a Xilinx device \cite{vitis_user_guide}.

\paragraph{Vitis AI}
The Vitis AI development environment consists of the Vitis AI development kit.
It contains an AI compiler, an AI quantizer, an AI optimizer as well as the libraries to run inference and the \acrfull{xrt}.
Example designs are also included.
AI inference on Edge devices and Alveo accelerator cards are available with Vitis AI \cite{vitis_ai_user_guide}.
Vitis AI replaces the old \acrshort{dnndk} toolchain since version 2019.2.

\paragraph{Acceleration Runtime}
% In the \acrfull{xrt} a combination of userspace and kernel driver components are included.
The \acrfull{xrt} contains a combination of userspace and kernel driver components.
% The advantage of the \acrshort{xrt} for the user is that there is very little porting effort when migrating an application from PCIe based platforms to MPSoC based edge platforms or from one edge platform to another.
The advantage of the \acrshort{xrt} for the user is that when migrating an application from PCIe-based platforms to MPSoC-based edge platforms or from one edge platform to another, there is very little porting effort.
When working with Vitis AI, it handles the \acrshort{xrt} in the background \cite{xrt_overview}.

\paragraph{Neural Network Runtime}
\acrshort{n2cube} is based on the \acrshort{xrt}.
It provides the \acrshortpl{api} to develop an \acrshort{ai} application in Python or C++ \cite{vitis_ai_user_guide}.

\paragraph{Versions}
Xilinx releases two to three versions per year.
% These follow the naming scheme, that first the year is written and then a number, which starts at 1 and is incremented for every new version.
These follow the naming scheme that the year is written first and then a number starting at 1, which is incremented with each new version.
It is important to use the same version for PetaLinux, \acrshort{xrt} and Vitis.
Vitis was first released in version 2019.2.

\paragraph{Using Xilinx Tools}
% While installing Vitis, Vivado is installed as well by default.
When Vitis is installed, Vivado is also installed by default.
The Xilinx tools are not added to the path when installed.
% Therefore, a shell script, located in the installation directory needs to be sourced.
Therefore, a shell script, which is located in the installation directory, must be sourced.
For example, sourcing PetaLinux could look like this:
\begin{lstlisting}[style=bash, caption={}, label=lst:source_tools]
  source /tools/petalinux/2019.2/settings.sh
\end{lstlisting}
% Avnet expects that the user installs the tools in the default install directory.
Avnet expects the user to install the tools in the default installation directory.
For Vitis the default directory is \texttt{/tools/Xilinx/} and for PetaLinux it is \texttt{/tools/}.
To avoid problems, it is recommended not to change the default directories.
% Install in default location, because of hardcoded paths.
