\chapter*{Abstract}
% Motivation
In a world of self-driving cars and automated quality control in manufacturing, real-time image classification is becoming increasingly important.
Artificial intelligence, and deep learning in particular, are achieving excellent classification accuracies, but there are some challenges.
% Problem statement
For one thing, high-resolution image acquisition systems require a lot of processing power.
For another, a large labeled dataset of training data is required to train deep convolutional neural networks.
% Approach
% A solution for the former is to use field-programmable gate arrays as hardware accelerators.
A solution for the former is to use field-programmable gate arrays (FPGAs) as hardware accelerators.
Therefore, an embedded system featuring a multiprocessor system-on-chip with an integrated FPGA is deployed.
The second problem is approached with data augmentation to artificially increase the size of the labeled dataset.
% Results
% The deployed convolutional neural network achieves a Top-1 accuracy of \SI{97.2}{\percent} and a Top-5 accuracy of \SI{99.5}{\percent}.
% The deployed convolutional neural network achieved a Top-1 accuracy of \SI{97.2}{\percent} and a Top-5 accuracy of \SI{99.5}{\percent}.
% This allows the deployed convolutional neural network to achieve a Top-1 accuracy of \SI{97.2}{\percent} and a Top-5 accuracy of \SI{99.5}{\percent}.
This allowed the deployed convolutional neural network to achieve a Top-1 accuracy of \SI{97.2}{\percent} and a Top-5 accuracy of \SI{99.5}{\percent}.
% The throughput of the image classification chain \SI{41.1}{fps}.
In addition, the throughput of the image classification chain reached \SI{41.1}{fps} for color images of $1280\times\SI{1024}{px}$.
% In addition, the throughput of the image classification chain reaches \SI{41.1}{fps} for color images of $1280\times\SI{1024}{px}$.
% Additionally, the throughput of the image classification chain equals \SI{41.1}{fps} for color images of $1280\times\SI{1024}{px}$.
% Conclusions
Using data augmentation significantly improved the real-world classification performance by reducing the impact of ambient light.
Furthermore, it completely eliminated the need to collect additional data samples.

% from https://users.ece.cmu.edu/~koopman/essays/abstract.html

\vspace{1cm}
\begin{tabular}{>{\bfseries}ll}
  Team  & Nico Canzani, Dominik M\"uller \\
  Keywords & Artificial Intelligence, FPGA, Convolutional Neural Network % Industrial Camera
\end{tabular}
