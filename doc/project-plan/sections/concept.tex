\chapter{Concept}
\label{ch:concept}

This chapter briefly explains the concept of this project.
The three main parts are the demonstrator, the training and the inference.

\section{Demonstrator}
\label{sec:demonstrator}

The demonstrator --- a throwing booth --- was not entirely finished during the previous project.
However, the things that still need to be done are not that time-consuming.
For this reason, those tasks are not included in the timetable.
They are carried out on the side to fill time gaps (e.g. during the training of a CNN model).

\section{Training}
\label{sec:training}

The development and training of the CNN model is an iterative process.
Various CNN architectures are experimented with and the results are compared to each other.
Furthermore, the labeled dataset allows for some experimentation as well.
For instance, each image has a label that indicates whether the object was fully or only partially visible.

\section{Inference}
\label{sec:inference}

A high-performance solution requires fast image acquisition and low-latency inference.
To implement the necessary components on the MPSoC with the integrated FPGA, a variety of toolchains and Linux-based operating systems can be used.
These options are initially examined carefully in order to select the most suitable one.

The specific implementation of the image acquisition and the inference depends on the overall performance and the available resources of the MPSoC development board.
For this reason, several performance tests are carried out to evaluate its real-time capabilities.

The final step is to verify the performance and accuracy of the entire system.
The accuracy is additionally compared to the accuracy of the computer solution.
